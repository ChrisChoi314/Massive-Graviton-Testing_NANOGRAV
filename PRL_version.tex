%\documentclass[prd,twocolumn,aps,psfig,showpacs,nofootinbib,nobibnotes,superscriptaddress,preprintnumbers,times]{revtex4}
\documentclass[prd,twocolumn,aps,psfig,nofootinbib,nobibnotes,superscriptaddress,preprintnumbers,times]{revtex4-2}
%\documentclass[prd,aps,psfig,nofootinbib,nobibnotes,superscriptaddress,preprintnumbers,times]{revtex4-2}\setlength{\topmargin}{-14mm}
\usepackage{graphicx,bm,color,amsmath,amssymb, mathtools,subcaption}
\graphicspath{{./fig/}}
\captionsetup{justification   = RaggedRight,
              singlelinecheck = false}

\usepackage[hidelinks]{hyperref}
%\hypersetup{
%  colorlinks   = true, %Colours links instead of ugly boxes
%  urlcolor     = blue, %Colour for external hyperlinks
%  linkcolor    = red, %Colour of internal links
%  citecolor    = blue %Colour of citations
%}
\hypersetup{
  colorlinks   = true, %Colours links instead of ugly boxes
  urlcolor     = [RGB]{46,48,146}, %Colour for external hyperlinks
  linkcolor    = [RGB]{46,48,146}, %Colour of internal links
  citecolor    = [RGB]{46,48,146} %Colour of citations
}

\def\red{\textcolor{red}}
\def\blue{\textcolor{blue}}

\usepackage{braket,orcidlink}

\usepackage{relsize}

%|||||||||||||||||||||||||||||||||||||||||||||||||||||||||||||||||||
%             Customized Commands
%|||||||||||||||||||||||||||||||||||||||||||||||||||||||||||||||||||
%  mathematical abbreviations
%
%
%
\newcommand{\BoldVec}[1]{\mathchoice%
  {\mbox{\boldmath $\displaystyle     #1$}}%
  {\mbox{\boldmath $\textstyle        #1$}}%
  {\mbox{\boldmath $\scriptstyle      #1$}}%
  {\mbox{\boldmath $\scriptscriptstyle#1$}}%
}
%\newcommand{\BoldVec}[1]{\bm{#1}}}
%
% math debs
\newcommand{\EQ}{\begin{equation}}
\newcommand{\EN}{\end{equation}}
\newcommand{\EQA}{\begin{eqnarray}}
\newcommand{\ENA}{\end{eqnarray}}
\newcommand{\eq}[1]{(\ref{#1})}
\newcommand{\EEq}[1]{Equation~(\ref{#1})}
\newcommand{\Eq}[1]{Eq.~(\ref{#1})}
\newcommand{\Eqs}[2]{Eqs~(\ref{#1}) and~(\ref{#2})}
\newcommand{\EEqs}[2]{Equations~(\ref{#1}) and~(\ref{#2})}
\newcommand{\eqs}[2]{(\ref{#1}) and~(\ref{#2})}
\newcommand{\Eqss}[2]{Eqs~(\ref{#1})--(\ref{#2})}
%\newcommand{\Sec}[1]{\S\,\ref{#1}}
%\newcommand{\Secs}[2]{\S\S\,\ref{#1} and~\ref{#2}}
\newcommand{\Sec}[1]{Sec.~\ref{#1}}
\newcommand{\Secs}[2]{Secs.~\ref{#1} and~\ref{#2}}
\newcommand{\App}[1]{Appendix~\ref{#1}}
\newcommand{\Fig}[1]{Fig.~\ref{#1}}
\newcommand{\FFig}[1]{Figure~\ref{#1}}
\newcommand{\Tab}[1]{Table~\ref{#1}}
\newcommand{\Figs}[2]{Figs.~\ref{#1} and \ref{#2}}
\newcommand{\Tabs}[2]{Tables~\ref{#1} and \ref{#2}}
%\newcommand{\bra}[1]{\langle #1\rangle}
\newcommand{\bbra}[1]{\left\langle #1\right\rangle}
\newcommand{\mean}[1]{\overline #1}
\newcommand{\meanB}{\overline{B}}
\newcommand{\meanC}{\overline{C}}
\newcommand{\meanU}{\overline{U}}
\newcommand{\meanW}{\overline{W}}
\newcommand{\meanPhi}{\overline{\Phi}}
\newcommand{\meanF}{\overline{\cal F}}
\newcommand{\meanR}{\overline{\cal R}}
\newcommand{\meanAA}{\overline{\bm{A}}}
\newcommand{\meanBB}{\overline{\bm{B}}}
\newcommand{\meanEE}{\overline{\bm{E}}}
\newcommand{\meanUU}{\overline{\bm{U}}}
\newcommand{\meanWW}{\overline{\bm{W}}}
\newcommand{\meanJJ}{\overline{\mbox{\boldmath $J$}}}
\newcommand{\meanuu}{\overline{\mbox{\boldmath $u$}}}
\newcommand{\meanGG}{\overline{\mbox{\boldmath $G$}}}
\newcommand{\meanAB}{\overline{\mbox{\boldmath $A\cdot B$}}}
\newcommand{\meanAoBo}{\overline{\mbox{\boldmath $A_0\cdot B_0$}}}
\newcommand{\meanApoBpo}{\overline{\mbox{\boldmath $A'_0\cdot B'_0$}}}
\newcommand{\meanApBp}{\overline{\mbox{\boldmath $A'\cdot B'$}}}
\newcommand{\meanuxB}{\overline{\mbox{\boldmath $\delta u\times \delta B$}}}
\newcommand{\meanemfs}{\overline{\cal E} {}}
\newcommand{\meanemf}{\overline{\mbox{\boldmath ${\cal E}$}} {}} %redundant
\newcommand{\meanAAAA}{\overline{\mbox{\boldmath ${\mathsf A}$}} {}}
\newcommand{\meanSSSS}{\overline{\mbox{\boldmath ${\mathsf S}$}} {}}
\newcommand{\meanAAA}{\overline{\mathsf{A}}}
\newcommand{\meanSSS}{\overline{\mathsf{S}}}
\newcommand{\meanCC}{\overline{\mbox{\boldmath ${\cal C}$}} {}}
\newcommand{\meanFF}{\overline{\mbox{\boldmath ${\cal F}$}} {}}
\newcommand{\meanRR}{\overline{\mbox{\boldmath ${\cal R}$}} {}}
\newcommand{\calFF}{\overline{\mbox{\boldmath ${\cal F}$}} {}}
\newcommand{\meanEMF}{\overline{\mbox{\boldmath ${\cal E}$}} {}}
\newcommand{\tildeFFFF}{\tilde{\mbox{\boldmath ${\cal F}$}}{}}{}
\newcommand{\hatFFFF}{\hat{\mbox{\boldmath ${\cal F}$}}{}}{}
\newcommand{\meanFFFF}{\overline{\mbox{\boldmath ${\cal F}$}}{}}{}
\newcommand{\meanFFF}{\overline{\cal F}}
\newcommand{\hatOO}{\hat{\bm{\Omega}}}
\newcommand{\hatAA}{\hat{\bm{A}}}
\newcommand{\hatBB}{\hat{\bm{B}}}
\newcommand{\tildeh}{\tilde{h}}
\newcommand{\tildeT}{\tilde{T}}
\newcommand{\tildehhh}{\tilde{\sf h}}
\newcommand{\tildeTTT}{\tilde{\sf T}}
%
% tilde
%
\newcommand{\eee}{{\sf e}}
\newcommand{\hhh}{{\sf h}}
\newcommand{\TTT}{{\sf T}}
\newcommand{\tildexx}{\tilde{\bm{x}}}
\newcommand{\tildeBB}{\tilde{\bm{B}}}
\newcommand{\tildeJJ}{\tilde{\bm{J}}}
\newcommand{\tildeA}{\tilde{A}}
\newcommand{\tildeB}{\tilde{B}}
\newcommand{\tildeJ}{\tilde{J}}
\newcommand{\tildeemf}{\tilde{\cal E}}
\newcommand{\teps}{\tilde{\epsilon} {}}
\newcommand{\tkapz}{\tilde{\kappa_0}}
\newcommand{\Oh}{\hat{\Omega}}
\newcommand{\zh}{\hat{z}}
\newcommand{\PC}{{\sc Pencil Code}~}
\newcommand{\PCS}{{\sc Pencil Code}}
%
%  unit vectors
%
\newcommand{\nullvector}{{\bf0}}
\newcommand{\nnn}{\hat{\mbox{\boldmath $n$}} {}}
\newcommand{\vvv}{\hat{\mbox{\boldmath $v$}} {}}
\newcommand{\rr}{\hat{\mbox{\boldmath $r$}} {}}
\newcommand{\xxx}{\hat{\mbox{\boldmath $x$}} {}}
\newcommand{\yyy}{\hat{\mbox{\boldmath $y$}} {}}
\newcommand{\zz}{\hat{\mbox{\boldmath $z$}} {}}
\newcommand{\pp}{\hat{\mbox{\boldmath $\phi$}} {}}
\newcommand{\ttt}{\hat{\mbox{\boldmath $\theta$}} {}}
\newcommand{\OOO}{\hat{\mbox{\boldmath $\Omega$}} {}}
\newcommand{\ooo}{\hat{\mbox{\boldmath $\omega$}} {}}
\newcommand{\BBBB}{\hat{\mbox{\boldmath $B$}} {}}
\newcommand{\kunit}{\hat{\mbox{$k$}} {}}
\newcommand{\nunit}{\hat{\mbox{$n$}} {}}
%
%  hatted quantities
%
\newcommand{\hatU}{\hat{U}}
\newcommand{\hatUU}{\hat{\bm{U}}}
%
%  vectors
%
\newcommand{\gggg}{\BoldVec{g} {}}
\newcommand{\ddd}{\BoldVec{d} {}}
\newcommand{\rrr}{\BoldVec{r} {}}
\newcommand{\xx}{\BoldVec{x}{}}
\newcommand{\yy}{\BoldVec{y} {}}
\newcommand{\zzz}{\BoldVec{z} {}}
\newcommand{\uu}{\BoldVec{u} {}}
\newcommand{\vv}{\BoldVec{v} {}}
\newcommand{\ww}{\BoldVec{w} {}}
\newcommand{\mm}{\BoldVec{m} {}}
\newcommand{\PP}{\BoldVec{P} {}}
\newcommand{\QQ}{\BoldVec{Q} {}}
\newcommand{\RR}{\BoldVec{R} {}}
\newcommand{\UU}{\BoldVec{U} {}}
\newcommand{\bb}{\BoldVec{b} {}}
\newcommand{\qq}{\BoldVec{q} {}}
\newcommand{\BB}{\BoldVec{B} {}}
\newcommand{\HH}{\BoldVec{H} {}}
\newcommand{\II}{\BoldVec{I} {}}
\newcommand{\AAA}{\BoldVec{A} {}}
\newcommand{\aaa}{\BoldVec{a} {}}
\newcommand{\aaaa}{\BoldVec{a} {}} %(convert aaa -> aaaa, compatibility problem)
%\newcommand{\eee}{\BoldVec{e} {}}
\newcommand{\jj}{\BoldVec{j} {}}
\newcommand{\JJ}{\BoldVec{J} {}}
\newcommand{\nn}{\BoldVec{n} {}}
\newcommand{\ee}{\BoldVec{e} {}}
\newcommand{\ff}{\BoldVec{f} {}}
\newcommand{\hh}{\BoldVec{h} {}}
\newcommand{\EE}{\BoldVec{E} {}}
\newcommand{\FF}{\BoldVec{F} {}}
\newcommand{\TT}{\BoldVec{T} {}}
\newcommand{\CC}{\BoldVec{C} {}}
\newcommand{\KK}{\BoldVec{K} {}}
\newcommand{\MM}{\BoldVec{M} {}}
\newcommand{\GG}{\BoldVec{G} {}}
\newcommand{\kk}{\BoldVec{k} {}}
\newcommand{\SSS}{\BoldVec{S} {}}
\newcommand{\grav}{\BoldVec{g} {}}
\newcommand{\nab}{\BoldVec{\nabla} {}}
\newcommand{\OO}{\BoldVec{\Omega} {}}
\newcommand{\oo}{\BoldVec{\omega} {}}
\newcommand{\LL}{\BoldVec{\Lambda} {}}
\newcommand{\llambda}{\BoldVec{\lambda} {}}
\newcommand{\pomega}{\BoldVec{\varpi} {}}
%
%  correlation tensors
%
\newcommand{\RRRR}{\bm{\mathsf{R}}}
\newcommand{\SSSS}{\bm{\mathsf{S}}}
\newcommand{\LLLL}{\mbox{\boldmath ${\sf L}$} {}}
\newcommand{\MMMM}{\bm{\mathsf{M}}}
\newcommand{\BBB}{\mbox{\boldmath ${\cal B}$} {}}
\newcommand{\emf}{\mbox{\boldmath ${\cal E}$} {}}
\newcommand{\FFF}{\mbox{\boldmath ${\cal F}$} {}}
\newcommand{\GGG}{\mbox{\boldmath ${\cal G}$} {}}
\newcommand{\HHH}{\mbox{\boldmath ${\cal H}$} {}}
\newcommand{\QQQ}{\mbox{\boldmath ${\cal Q}$} {}}
%
%  operators  (roman)
%
\newcommand{\ii}{{\rm i}}
\newcommand{\grad}{{\rm grad} \, {}}
\newcommand{\curl}{{\rm curl} \, {}}
\newcommand{\dive}{{\rm div}  \, {}}
\newcommand{\Dive}{{\rm Div}  \, {}}
\newcommand{\diag}{{\rm diag}  \, {}}
\newcommand{\DD}{{\rm D} {}}
\newcommand{\dd}{{\rm d} {}}
\newcommand{\const}{{\rm const}  {}}
\newcommand{\crit}{{\rm crit}  {}}
\def\degr{\hbox{$^\circ$}}
\def\la{\mathrel{\mathchoice {\vcenter{\offinterlineskip\halign{\hfil
$\displaystyle##$\hfil\cr<\cr\sim\cr}}}
{\vcenter{\offinterlineskip\halign{\hfil$\textstyle##$\hfil\cr<\cr\sim\cr}}}
{\vcenter{\offinterlineskip\halign{\hfil$\scriptstyle##$\hfil\cr<\cr\sim\cr}}}
{\vcenter{\offinterlineskip\halign{\hfil$\scriptscriptstyle##$\hfil\cr<\cr\sim\cr}}}}}
\def\ga{\mathrel{\mathchoice {\vcenter{\offinterlineskip\halign{\hfil
$\displaystyle##$\hfil\cr>\cr\sim\cr}}}
{\vcenter{\offinterlineskip\halign{\hfil$\textstyle##$\hfil\cr>\cr\sim\cr}}}
{\vcenter{\offinterlineskip\halign{\hfil$\scriptstyle##$\hfil\cr>\cr\sim\cr}}}
{\vcenter{\offinterlineskip\halign{\hfil$\scriptscriptstyle##$\hfil\cr>\cr\sim\cr}}}}}
%
%  numbers
%
\def\Ta{\mbox{\rm Ta}}
\def\Ra{\mbox{\rm Ra}}
\def\Ma{\mbox{\rm Ma}}
\def\Sh{\mbox{\rm Sh}}
\def\Roo{\mbox{\rm Ro}^{-1}}
\def\Pra{\mbox{\rm Pr}}
\def\Pran{\mbox{\rm Pr}}
\def\Pm{\mbox{\rm Pr}_{\rm M}}
\def\Rm{\mbox{\rm Re}_{\rm M}}
\def\Rey{\mbox{\rm Re}}
\def\Imag{\mbox{\rm Im}}
\def\Pe{\mbox{\rm Pe}}
\def\epsK{\epsilon_{\rm K}}
\def\epsM{\epsilon_{\rm M}}
\def\EEi{{\cal E}_i}
\def\EEK{{\cal E}_{\rm K}}
\def\EEM{{\cal E}_{\rm M}}
\def\EEKM{{\cal E}_{\rm K/M}}
\def\EEGW{{\cal E}_{\rm GW}}
\def\OmK{{\Omega}_{\rm K}}
\def\OmM{{\Omega}_{\rm M}}
\def\OMG  W{{\Omega}_{\rm GW}}
\def\hrms{{h}_{\rm rms}}
\def\EEtot{{\cal E}_{\rm tot}}
\def\EErad{{\cal E}_{\rm rad}}
\def\EElam{{\cal E}_\lambda}
\def\EEcrit{{\cal E}_{\rm crit}}
\def\HHGW{{\cal H}_{\rm GW}}
\def\HHK{{\cal H}_{\rm K}}
\def\HHM{{\cal H}_{\rm M}}
\def\EGW{E_{\rm GW}}
\def\HGW{H_{\rm GW}}
\def\EK{E_{\rm K}}
\def\EM{E_{\rm M}}
\def\HM{H_{\rm M}}
\def\hc{h_{\rm c}}
\def\cs{c_{\rm s}}
\def\xiM{\xi_{\rm M}}
\def\xiK{\xi_{\rm K}}
\def\kf{k_{\rm f}}
%\def\kf{k_\ast}
\def\vA{v_{\rm A}}
\def\urms{u_{\rm rms}}
\def\Urms{U_{\rm rms}}
\def\Brms{B_{\rm rms}}
\def\kappaOO{\kappa_{\Omega\Omega}}
\def\kappaO{\kappa_{\Omega}}
\def\kappat{\kappa_{\rm t}}
\def\kappatz{\kappa_{\rm t0}}
\def\nut{\nu_{\rm t}}
\def\etatz{\eta_{\rm t0}}
\def\etat{\eta_{\rm t}}
\def\etaT{\eta_{\rm T}}
\def\Beq{B_{\rm eq}}
\def\tmax{t_{\max}}
%
\newcommand{\ea}{{\em et al. }}
\newcommand{\eaa}{{\em et al. }}
\def\half{{\textstyle{1\over2}}}
\def\threehalf{{\textstyle{3\over2}}}
\def\onethird{{\textstyle{1\over3}}}
\def\twothird{{\textstyle{2\over3}}}
\def\fourthird{{\textstyle{4\over3}}}
\def\quarter{{\textstyle{1\over4}}}
%
\newcommand{\W}{\,{\rm W}}
\newcommand{\V}{\,{\rm V}}
\newcommand{\kV}{\,{\rm kV}}
\newcommand{\eV}{\,{\rm eV}}
\newcommand{\MeV}{\,{\rm MeV}}
\newcommand{\GeV}{\,{\rm GeV}}
\newcommand{\T}{\,{\rm T}}
\newcommand{\uG}{\,\mu{\rm G}}
\newcommand{\G}{\,{\rm G}}
\newcommand{\Hz}{\,{\rm Hz}}
\newcommand{\mHz}{\,{\rm mHz}}
\newcommand{\nHz}{\,{\rm nHz}}
\newcommand{\uHz}{\,\mu{\rm Hz}}
\newcommand{\kHz}{\,{\rm kHz}}
\newcommand{\kG}{\,{\rm kG}}
\newcommand{\K}{\,{\rm K}}
\newcommand{\g}{\,{\rm g}}
\newcommand{\s}{\,{\rm s}}
\newcommand{\ms}{\,{\rm ms}}
\newcommand{\cm}{\,{\rm cm}}
\newcommand{\m}{\,{\rm m}}
\newcommand{\km}{\,{\rm km}}
\newcommand{\kms}{\,{\rm km/s}}
\newcommand{\kg}{\,{\rm kg}}
\newcommand{\Mm}{\,{\rm Mm}}
\newcommand{\pc}{\,{\rm pc}}
\newcommand{\kpc}{\,{\rm kpc}}
\newcommand{\Mpc}{\,{\rm Mpc}}
\newcommand{\yr}{\,{\rm yr}}
\newcommand{\Myr}{\,{\rm Myr}}
\newcommand{\Gyr}{\,{\rm Gyr}}
\newcommand{\erg}{\,{\rm erg}}
\newcommand{\mol}{\,{\rm mol}}
\newcommand{\dyn}{\,{\rm dyn}}
\newcommand{\J}{\,{\rm J}}
\newcommand{\RM}{\,{\rm RM}}
\newcommand{\AU}{\,{\rm AU}}
\newcommand{\A}{\,{\rm A}}
%
\def\hX{h_\times}
\def\hT{h_+}
\def\thT{\tilde{h}_+}
\def\thX{\tilde{h}_\times}
\def\dhT{\dot{h}_+}
\def\dhX{\dot{h}_\times}
\def\dhhT{\dot{\hat{h}}_+}
\def\dhhX{\dot{\hat{h}}_\times}
\def\dhhTX{\dot{\hat{h}}_{+/\times}}
\def\dthT{\dot{\tilde{h}}_+}
\def\dthX{\dot{\tilde{h}}_\times}
\def\dthTX{\dot{\tilde{h}}_{+/\times}}
%
%  journals
%
\newcommand{\arXiv}[3]{, ``#3,'' arXiv:#2 (#1).}
\newcommand{\yjcap}[3]{, J.\ Cosmol.\ Astropart.\ Phys. {\bf #2} (#1) #3.}
\newcommand{\yjas}[3]{, J. Atmosph. Sci. {\bf #2}, #3 (#1).}
\newcommand{\yan}[3]{, Astron. Nachr. {\bf #2}, #3 (#1).}
\newcommand{\yact}[3]{, Acta Astron. {\bf #2}, #3 (#1).}
\newcommand{\yana}[3]{, Astron. Astrophys. {\bf #2}, #3 (#1).}
\newcommand{\yanas}[3]{, Astron. Astrophys. Suppl. {\bf #2}, #3 (#1).}
\newcommand{\yanal}[3]{, Astron. Astrophys. Lett. {\bf #2}, #3 (#1).}
\newcommand{\yass}[3]{, Astrophys. Spa. Sci. {\bf #2}, #3 (#1).}
\newcommand{\ysci}[3]{, Science {\bf #2}, #3 (#1).}
\newcommand{\ysph}[3]{, Solar Phys. {\bf #2}, #3 (#1).}
\newcommand{\yjetp}[3]{, Sov. Phys. JETP {\bf #2}, #3 (#1).}
\newcommand{\yspd}[3]{, Sov. Phys. Dokl. {\bf #2}, #3 (#1).}
\newcommand{\ysov}[3]{, Sov. Astron. {\bf #2}, #3 (#1).}
\newcommand{\ysovl}[3]{, Sov. Astron. Lett. {\bf #2}, #3 (#1).}
\newcommand{\ymn}[3]{, Mon.\ Not.\ R.\ Astron.\ Soc.\ {\bf #2}, #3 (#1).}
\newcommand{\ymhd}[3]{, Magnetohydrohydrodyn. {\bf #2}, #3 (#1).}
\newcommand{\yqjras}[3]{, Quart. J. Roy. Astron. Soc. {\bf #2}, #3 (#1).}
\newcommand{\ynat}[3]{, Nature {\bf #2}, #3 (#1).}
\newcommand{\yjfm}[4]{, ``#4,'' J. Fluid Mech. {\bf #2}, #3 (#1).}
\newcommand{\pjfm}[1]{, J. Fluid Mech., in press (#1).}
\newcommand{\sjfm}[1]{, J. Fluid Mech., submitted (#1).}
\newcommand{\ypr}[3]{, Phys.\ Rev.\ {\bf #2}, #3 (#1).}
\newcommand{\yprd}[4]{, ``#4,'' Phys.\ Rev.\ D {\bf #2}, #3 (#1).}
\newcommand{\ypre}[3]{, Phys.\ Rev.\ E {\bf #2}, #3 (#1).}
\newcommand{\yprf}[4]{, ``#4,'' Phys.\ Rev.\ Fluids {\bf #2}, #3 (#1).}
\newcommand{\yprl}[4]{, ``#4,'' Phys.\ Rev.\ Lett.\ {\bf #2}, #3 (#1).}
\newcommand{\yphl}[3]{, Phys.\ Lett.\ {\bf #2}, #3 (#1).}
\newcommand{\pprl}[1]{, Phys. Rev. Lett., in press (#1).}
\newcommand{\yepl}[3]{, Europhys. Lett. {\bf #2}, #3 (#1).}
\newcommand{\pcsf}[2]{, Chaos, Solitons \& Fractals, in press (#1).}
\newcommand{\ycsf}[3]{, Chaos, Solitons \& Fractals{\bf #2}, #3 (#1).}
\newcommand{\yprs}[3]{, Proc. Roy. Soc. Lond. {\bf #2}, #3 (#1).}
\newcommand{\yptrs}[3]{, Phil. Trans. Roy. Soc. {\bf #2}, #3 (#1).}
\newcommand{\yptrsa}[4]{, ``#4,'' Phil. Trans. Roy. Soc. Lond. A, {\bf #2}, #3 (#1).}
\newcommand{\yjcp}[3]{, J. Comp. Phys. {\bf #2}, #3 (#1).}
\newcommand{\yjgr}[3]{, J. Geophys. Res. {\bf #2}, #3 (#1).}
\newcommand{\ygrl}[3]{, Geophys. Res. Lett. {\bf #2}, #3 (#1).}
\newcommand{\yobs}[3]{, Observatory {\bf #2}, #3 (#1).}
\newcommand{\yaj}[3]{, Astronom. J. {\bf #2}, #3 (#1).}
\newcommand{\sapj}[3]{, ``#3,'' Astrophys. J., submitted, arXiv:#2  (#1).}
\newcommand{\papj}[3]{, ``#3,'' Astrophys. J., in press, arXiv:#2  (#1).}
\newcommand{\yapj}[4]{, ``#4,'' Astrophys. J. {\bf #2}, #3 (#1).}
\newcommand{\yapjs}[3]{, Astrophys. J. Suppl. {\bf #2}, #3 (#1).}
\newcommand{\yapjl}[3]{, Astrophys. J. {\bf #2}, #3 (#1).}
\newcommand{\ycqg}[3]{, Class. Quant. Grav. {\bf #2}, #3 (#1).}
\newcommand{\ypp}[3]{, Phys. Plasmas {\bf #2}, #3 (#1).}
\newcommand{\yppcf}[3]{, Plasmas Phys. Contr. Fusion {\bf #2}, #3 (#1).}
\newcommand{\ppp}[1]{, Phys. Plasmas, in press (#1).}
\newcommand{\ypasj}[3]{, Publ. Astron. Soc. Japan {\bf #2}, #3 (#1).}
\newcommand{\ypac}[3]{, Publ. Astron. Soc. Pacific {\bf #2}, #3 (#1).}
\newcommand{\yaraa}[3]{, Ann. Rev. Astron. Astrophys. {\bf #2}, #3 (#1).}
\newcommand{\yanar}[3]{, Astron. Astrophys. Rev. {\bf #2}, #3 (#1).}
\newcommand{\yanp}[3]{, Ann. Phys. {\bf #2}, #3 (#1).}
\newcommand{\yanf}[3]{, Ann. Rev. Fluid Dyn. {\bf #2}, #3 (#1).}
\newcommand{\ypf}[4]{, ``#4,'' Phys. Fluids {\bf #2}, #3 (#1).}
\newcommand{\yphy}[3]{, Physica {\bf #2}, #3 (#1).}
\newcommand{\ygafd}[4]{, ``#4,'' Geophys. Astrophys. Fluid Dyn. {\bf #2}, #3 (#1).}
\newcommand{\yrpp}[3]{, Rep. Prog. Phys. {\bf #2}, #3 (#1).}
\newcommand{\yptp}[3]{, Progr. Theor. Phys. {\bf #2}, #3 (#1).}
\newcommand{\yjour}[5]{, ``#5,'' #2 {\bf #3}, #4 (#1).}
\newcommand{\pjour}[3]{, #2, in press (#1).}
\newcommand{\sjour}[3]{, #2, submitted (#1).}
\newcommand{\yprep}[2]{, #2, preprint (#1).}
\newcommand{\pproc}[3]{, (ed. #3), #2 (#1) (to appear).}
\newcommand{\yproc}[4]{, (ed. #4), pp. #2. #3 (#1).}
\newcommand{\ybook}[3]{, {\em #2}. #3 (#1).}
\newcommand{\neff}{N_{\rm eff}}
\newcommand{\dneff}{\Delta N_{\rm eff}}
\newcommand{\neffv}{N_{\rm eff}^{(\nu)}}

\newcommand{\inv}{\rm inv}

\usepackage{braket}

\begin{document}

\title{Do Pulsar Timing Datasets Favor Massive Gravity?}

\date{\today}
\author{Chris~Choi\,\orcidlink{0009-0005-2328-3044}}
\email{Contact author: minyeonc@andrew.cmu.edu}
\affiliation{McWilliams Center for Cosmology and Astrophysics and Department of Physics, \href{https://ror.org/05x2bcf33}{Carnegie Mellon University}, Pittsburgh, Pennsylvania 15213, USA}

\author{Tina~Kahniashvili\,\orcidlink{0000-0003-0217-9852}}
\email{Contact author: tinatin@andrew.cmu.edu}
\affiliation{McWilliams Center for Cosmology and Astrophysics and Department of Physics, \href{https://ror.org/05x2bcf33}{Carnegie Mellon University}, Pittsburgh, Pennsylvania 15213, USA}
\affiliation{School of Natural Sciences and Medicine, \href{https://ror.org/051qn8h41}{Ilia State University}, 0194 Tbilisi, Georgia}
\affiliation{\href{https://ror.org/02gkgrd84}{Abastumani Astrophysical Observatory}, Tbilisi GE-0179, Georgia}

\begin{abstract}

There are several observational phenomena suggesting that the standard model of cosmology and particle physics needs to be revised. We consider one extension of general relativity in order to address this: massive gravity (MG). In this Letter, we explore the imprints of MG on the propagation of gravitational waves (GWs): their modified dispersion relation and their additional (scalar- and two-vector) polarization modes on the stochastic GW background (SGWB) detected by pulsar timing arrays (PTAs). We analyze the effects of massive GWs on the Helling-Downs curve induced by modification of the overlap reduction function. Our study consists of analyzing observational data from the NANOGrav 15-year dataset and the Chinese PTA Data Release I, and is independent of the origin of the SGWB (astrophysical or cosmological).  
By considering the bounds of the graviton mass imposed through the dispersion relation, 
we scrutinize the possibility of detecting traces of MG in the PTA observational data.  
We find that massive GWs predict better fits for the observed pulsar correlations. Future PTA missions with more precise data will hopefully be able to detect the GW additional polarization modes and might be effectively used to constrain the graviton mass.
\end{abstract}

\maketitle

%\section{Introduction}
\textit{Introduction}---The standard cosmological concordance model assumes that general relativity (GR) is the correct theory of gravity on cosmological length and time scales, and that the acceleration of the universe is due to a cosmological constant ($\Lambda$) that has a time-independent energy density and becomes dominant at late times \cite{Dodelson:2020bqr}. 
Despite the success of the concordance model, there are several unanswered questions, including the gauge hierarchy problem, the smallness of gravity compared to the other standard model forces, and the failure to formulate a unified theory of quantum gravity, among many others \cite{Dvali:2013qwe, Moffat:1998vi}. 
These puzzles suggest that GR is not suitable for describing physical processes at cosmological scales, and thus one possibility is to assume that the true theory of gravity differs from GR \cite{deRham:2023byw}. 

%%%%%%%%%%%%%%%%%%%%%%%%%%%%%%%%%%%%%%%%%%%%%%%%%%%%%%%%%%%%%%%%%%%%%%%%%%%%%%%%%%%%%%%%%%%%%%%

One of the most active areas of research in gravity theory stems from the assumption that the graviton has a non-zero mass $m_g$, in a theory known as massive gravity (MG). It may seem at first like a strange assumption, but there is no reason why the mass of the force carrier of gravity must be 0. The idea of a nonzero graviton mass has a long history, starting from the formulation of MG at the linear level by Fierz and Pauli \cite{Fierz:1939ix} in the 1930s. In contrast to GR, where gravitational waves (GWs) possess just two degrees of freedom helicity (spin) $\pm 2$, tensor modes), generic MG theories have an additional three degrees of freedom, namely helicity (spin) -$0$ (scalar) and $\pm 1$ (vector) modes. As a result, MG is subject to van Dam-Veltman-Zakharov (vDVZ) discontinuity \cite{vanDam:1970vg,Zakharov:1970cc}, where the five modes of MG do not reduce to the two modes that we expect in GR in the massless limit as $m_g \rightarrow 0$. 
Taking into account the nonlinear effects of the strong gravitational potential, the Vainshtein mechanism \cite{Vainshtein:1972sx} ensures the screening of additional scalar and vector modes, and correspondingly frees MG from the vDVZ discontinuity\footnote{GR is recovered in a strong gravitational field \cite{Tasinato:2013rza}, allowing us to verify GR in terrestrial and solar system level tests.}.

The effects of MG appear only on cosmological scales, possibly leading to an accelerated expansion \cite{DAmico:2011eto}.
However, extensions in the nonlinear regime are unsatisfactory due to the presence of an unhealthy sixth ``ghost'' mode\footnote{Until 2010, it was thought that all Lorentz-invariant MG theories were characterized by the unhealthy presence of the ghost-mode, and thus were not valid \cite{deRham:2010kj}.} \cite{Boulware:1972yco}.
More recently, groundbreaking progress has been made through the formulation of ghost-free MG in the deRham-Gabadadze-Tolley (dRGT) theory \cite{deRham:2010ik,deRham:2010kj}, and its bigravity generalization \cite{Hassan:2011zd} (see Ref.\ \cite{deRham:2023ngf} for a review). Since then, many different modifications of dRGT and bigravity have been proposed \cite{Hinterbichler:2011tt,deRham:2014zqa,Koyama:2015vza,deRham:2016nuf,Hinterbichler:2016try, Cusin:2016ytz, Kenna-Allison:2019tbu, Kazempour:2022giy},  
leading to investigations of the consequences of MG and massive cosmologies (pioneered by Ref. \cite{DAmico:2011eto} and further explored by Refs. \cite{Gratia:2012wt,Gumrukcuoglu:2012aa,Maeda:2013bha,Akrami:2013pna,Zhang:2013noa,Lambiase:2012fv,Koyama:2011wx,Tasinato:2012ze, Solomon:2014iwa, Akrami:2013ffa,Koennig:2014ods,Gumrukcuoglu:2016hic, Heisenberg:2024uwq, Smirnov:2025yru}),
such as in the context of the generation and propagation of GW \cite{DeFelice:2013awa,Gumrukcuoglu:2013nza,DeFelice:2013bxa,DeFelice:2015moy,Babichev:2015xha,Sakstein:2017bws}. 

The direct detection of GWs at high frequencies by interferometers \cite{LIGOScientific:2016aoc,LIGOScientific:2016sjg, KAGRA:2020agh, VIRGO:2014yos} and the more recent detection of the stochastic GW background (SGWB) through various PTA collaborations \cite{Agazie:2023, Xu:2023wog,EPTA:2023sfo,EPTA:2023akd,EPTA:2023fyk, Zic:2023gta,Reardon:2023gzh}, have ignited further interest in GW astronomy, giving us the unique possibility of constraining fundamental physics and understanding the nature of gravity. 
%TK changed a bit - please check above 
In our study, we use data collected by two GW detection missions, such as NANOGrav \cite{Agazie:2023} and the Chinese PTA (CPTA) \cite{Xu:2023wog}, chosen for their unique approach to their data, to help us better understand gravity and connect MG to observables. 
%TK maybe add LIGO also - just addressing both of them? for LIGO GW and PTA SGWB 
%CC added line about ligo and other interferometers
%TK maybe LIGo shou;d go first?
%CC swapped the order


In this Letter, we derive the overlap reduction function (ORF) and the dispersion relation in the case of ghost-free MG, taking into account the effect of these extra polarization modes. We carefully calculate, first analytically and then numerically, the contribution of the modes to the ORF. 
This work differs from previous work \cite{Liang:2021bct, Anholm:2008wy, Arjona:2024cex, Lee:2013awh} because of the particular emphasis we place on the non-suppression of the exponential factors. This is only possible because we analyze the ORFs in the context of an optimistic PTA observation prospect. In what follows, we will be using natural units, setting $c = $$\ \hbar = $$\ k_B = $$\ 1$.

%\section{Setup}\label{sec:setup}

\textit{Massive Gravity}---
The following effects of MG on GWs can phenomenologically be used to constrain the graviton mass \cite{deRham:2016nuf}:

(i) \textit{Yukawa-like exponential suppression}---Yukawa suppression is expected when the force carrier possesses a nonzero mass. In the case of MG, suppression of the gravitational potential is of the form $\Phi \propto e^{-m_gR}/R$, dependent on the length scale $R$. There is also a suppression of GWs at wavelengths larger than the Compton length scale of the graviton ($R_g \simeq m_g^{-1}$)\footnote{The length scale determined by the graviton mass through the Yukawa suppression can naively be treated as a ``gravitational causal horizon'', outside of which we expect the correlations of gravitational perturbations to be suppressed \cite{Will:1997bb}.}.

(ii) \textit{modification of the GW dispersion relation}---The dispersion relation gains a massive term. This implies a difference between the GW propagation speed ($c_g$) and the speed of light, which can be used to constrain $m_g$ \cite{LIGOScientific:2017vwq, LIGOScientific:2017zic, LIGOScientific:2017ync}. 

(iii) \textit{additional degrees of polarization}---These are characterized by the additional helicity-$0$ and helicity-$\pm 1$ modes. The main consequence of the helicity-$0$ polarization for the graviton is the coupling of the helicity-0 mode to matter (the so-called {\it fifth force}) \cite{deRham:2014naa}. This would affect the evolution of density perturbations and the growth of cosmological structures \cite{Sipp:2022kmb}. Some theories of massive gravity, such as the minimal theory of massive gravity \cite{DeFelice:2015hla}, only have two tensor modes, as in GR, but a general theory of massive gravity that assumes Lorentz invariance will have five in total.

A resounding confirmation of the graviton's non-zero mass will involve the observation of all three of these effects at the corresponding scales. In this Letter, we shall turn our attention to the latter two effects, (ii) and (iii). Gravitational waves are described by perturbations to the
%TK maybe on top or above? 
metric tensor, $g_{\mu\nu} = \eta_{\mu\nu} + h_{\mu\nu}$.
%TK add that you use \mu\nu for 1-4 and i, j, 1-3 go to the spatial components. I am not sure if we should use transverse-traceless projection - check Emir's paper how he defines
%CC i reviewed his paper and it seems like he does use the transverse traceless projection. should i use his formalism for the metric and ds^2?, also found this writeup going into the formalism regarding how to express the metric https://wwwusers.ts.infn.it/~milotti/Didattica/GravitationalWaves/handouts/13-TTGauge.pdf but wasnt sure which one to use
Here, we use the (+,--,--,--) convention for the Minkowski metric $\eta_{\mu\nu}$, and the Greek indices $\mu,\nu$ indicate a summation over all four of the spacetime coordinates. Whenever the region of observation is much smaller than the radius of curvature induced by the gravitational wave, we may express $h_{\mu\nu}$ as a plane wave \cite{Isi:2018miq}
\begin{equation}\label{eqn:planewave}
    h_{\mu\nu}(x) = \frac{1}{(2\pi)^4}\int d\boldsymbol{k} \ h_{\mu\nu}(k) e^{ikx} \ .
\end{equation}
Here, we have the 4-dimensional measure $d\boldsymbol{k} \equiv d^4 k 2\delta(|\boldsymbol{k}|^2 - |k_{\omega}|^2)/|\boldsymbol{k}|$ where $k_{\omega} \equiv \boldsymbol{k}(\omega)$ and is given by the dispersion relation. The dispersion relation can be derived directly from the relativistic energy-momentum relation, $E^2 = m_g^2 + |{\boldsymbol{p}}|^2$, by noting that $E = \omega$, the angular frequency of the GW, and ${\boldsymbol{p}} = \boldsymbol{k}$, the wavevector of the GW. This yields the following expression for the dispersion relation: 
\begin{equation}\label{eq:dispersion}
    \omega^2 = |\boldsymbol{k}|^2+ m_g^2 \ ,
\end{equation}
where $\omega \equiv k_0 = 2\pi f$, the frequency of the GW. If the graviton were massless, then we would have the standard dispersion relation, $\omega^2 = |\boldsymbol{k}|^2$, where the angular frequency is equivalent to the wavenumber, but the dispersion relation gains a nontriviality due to the nonzero mass in MG. 
% define f as frequency, k, wavenumber, say about non massive case, k^2\Omega^2, 
%CC defined and discussed now
To build intuition for this, we remind ourselves that an identical dispersion relation is obtained from the Proca equations \cite{Proca:1936fbw}, which describe massive helicity (spin) $\pm$ 1 particles. A careful derivation shows how the mass term emerges \cite{Wang:2024kir} for the massive photon. It is therefore not unexpected that the dispersion relation for the massive graviton, in the context of a helicity (spin) $\pm 2$ field, takes on the same form. 

In general, the dispersion relation is dependent on the scale factor $a$ and the propagation speed of gravitational waves $c_g$ \cite{Gumrukcuoglu:2012wt}. We assume that $a=1$, since we are observing the gravitational waves at present, and $c_g \leq  1$, based on observational constraints \cite{LIGOScientific:2017vwq, LIGOScientific:2017zic, LIGOScientific:2017ync}.
The dispersion relation provides a lower bound on the frequency for a GW, given by the mass of the graviton $m_g$, provided that $k$ goes to 0. In other words, $\omega$ must be greater than or equal to $m_g$; a departure from the behavior of GWs in GR.

\textit{Overlap reduction function}---Suppose there is a massive GW propagating in an arbitrary spatial direction specified by the unit vector $\hat{\Omega} = (\sin\theta \cos\varphi,
                        \sin\theta \sin\varphi,
                        \cos\theta)$.
This GW affects the pulsar timing measurements by perturbing the pulsar signals, originally emitted with a constant frequency $\nu_0$, as they travel from their source pulsars to our telescopes. In the barycenter reference frame of the solar system, the frequency shift is characterized by the redshift $z(t) = (\nu_0 - \nu(t))/\nu_0$. PTA collaborations actually measure the integral of $z(t)$, otherwise known as the anomalous residual, 
%TK reference, 
but it is trivial to rewrite these equations in terms of the residual if we so desire. We take advantage of the freedom of choice granted by their simple relation, and we proceed with the redshift. For PTA data, a quantity of interest is the two-point correlation of the total redshift, $\langle \tilde{z}(f) \tilde{z}(f') \rangle$, where $\tilde{z}(f)$ is the sky-averaged redshift $\tilde{z}(f) \equiv \int d^2\hat{\bf \Omega} z(f, \hat{\bf \Omega})$. The expression for the two-point correlation is given by
\begin{equation}\label{eqn:two_point_z}
    \langle \tilde{z}(f) \tilde{z}(f') \rangle = \frac{3H_0^2\delta^2\left(\hat{\bf \Omega} , \hat{\bf \Omega}'\right)\delta_{ii'}\delta(f - f')}{32 \beta \pi^3|f|^3}\Omega_{\text{gw}}(|f|) \Gamma(|f|).
\end{equation}
Here, $\beta$ is the normalization factor introduced so that $\Gamma(|f|) = \frac{1}{2}$ for coaligned coincident pulsars to match the conventions of the Hellings-Downs (HD) correlation \cite{Romano:2023zhb}. $\Omega_{\text{gw}}(|f|)$ is the energy density for the GWs. Details of this derivation can be found in Refs.~\cite{Anholm:2008wy, Liang:2021bct}, which ultimately derive from even earlier work \cite{Detweiler:1979wn, Estabrook:1975jtn, Kaufmann:1970}. A wealth of literature has been produced in the hopes of analyzing the behavior of the energy density in MG in the context of PTAs \cite{Choi:2023tun, Wu:2023rib, Kenjale:2024rsc, He:2021bqm}, but we turn our attention to the last factor in the expression: the ORF $\Gamma(|f|)$. 

We assume that there are two pulsars, located at distances $L_1$ and $L_2$ from the solar-system barycenter, and in directions $\hat{p}_1$ and $\hat{p}_2$. For convenience, we assume pulsar 1 is situated along the $z$-axis and pulsar 2 is situated on the $x-z$ plane, an angle $\xi$ away from pulsar 1. We define the ORF for each polarization type
\begin{equation}\label{eq:orf_type}
    \Gamma_I(|f|) = \mathlarger{\sum}_{i \in I} \int_{S^2} d^2 \hat{\bf{\Omega}} \mathcal{E}_1(-f, \hat{\Omega}) \mathcal{E}_2(f, \hat{\Omega}) F_1^{(i)}(\hat{\bf{\Omega}}) F_2^{(i)}(\hat{\bf{\Omega}}) \ ,
\end{equation}

where $I = \{T,V,S\}$ for each type of polarization, and we have defined the receiving function $F^{(i)}_j(\hat{\bf \Omega})$ and the exponential factor $\mathcal{E}_j(f, \hat{\Omega})$ as 
\begin{equation}\label{eqn:recieving}
    \begin{aligned}
        F^{(i)}_j(\hat{\Omega}) &= \frac{\hat{p}^\mu_j}{2}\left(-\frac{\hat{p}^\nu_j}{1+\frac{|\boldsymbol{k}|}{k_0} \hat{\boldsymbol{\Omega}} \cdot \hat{{\boldsymbol{p}}_j}} \epsilon_{\mu \nu}^{(i)}+\epsilon_{0 \mu}^{(i)}\right) , \\ 
        \mathcal{E}_j(f, \hat{\Omega}) &= e^{-i2\pi f L_j\left( 1 + \frac{|\boldsymbol{k}|}{k_0} \hat{\bf \Omega}\cdot \hat{\boldsymbol{p}}_j\right)} - 1 \ . 
    \end{aligned}
\end{equation}
The scalar product $\hat{\bf \Omega}\cdot \hat{\boldsymbol{p}}_j$ comes directly from applying the delta function to the complex exponential in the plane-wave decomposition of the metric perturbation (\ref{eqn:planewave}). 
It is possible to decompose the ORF into different modes because the tensor, vector, and scalar modes decouple from each other with an appropriate gauge-fixing term, allowing the kinetic terms to fully diagonalize \cite{Hinterbichler:2011tt}. Our effective overlap reduction function is given by 
\begin{equation}\label{eq:eff_orf}
    \tilde{\Gamma}_{T} = \beta \left(\Gamma_{T} + \Gamma_{V} \frac{\Omega_V}{\Omega_T} + \Gamma_{S} \frac{\Omega_S}{\Omega_T} \right) \ ,
\end{equation}
which behaves as an effective tensor mode, since PTA observations are not capable of distinguishing the different polarization modes from each other. In general, the energy densities $\Omega_T, \Omega_V,$ and $\Omega_S$ are frequency dependent, but we will suppress the frequency dependence in our analysis and assume that the energy densities of each polarization mode are equipartitioned, that is, $\Omega_T = \Omega_V = \frac{1}{2}\Omega_S$. 

\textit{Observational constraints}---Pulsar timing arrays are sensitive to frequencies in the range \cite{Moore:2014lga}
\begin{equation}\label{eq:freqrange}
    \frac{1}{T_{\text{obs}}} < f < \frac{1}{\delta t}\ ,
\end{equation} 
%$1/T_{\text{obs}} < f < 1/\delta t,$
where $T_{\text{obs}}$ is the total length of time the pulsars are observed and $\delta t$ is the cadence, that is, how often the pulsars are measured. For reference, $T_{\text{obs}} \approx$ 15 years and $\delta t \approx$ 0.5 weeks for the most frequent observations of a pulsar observed in NANOGrav. 
In the best-case scenario of PTA observation, we would be able to observe pulsars for nearly a century, if not longer. This corresponds to a frequency of $f_{\text{min}} \sim 3.17 \times 10^{-10} \Hz$, setting the lower limit of our frequency range. The closest pulsars used for PTAs are about $L_{\text{min}} \sim 100$ ly away \cite{Anholm:2008wy}, yielding $fL \sim 1$. Therefore, we do not suppress the exponential factors, unlike in Refs. \cite{Liang:2021bct,Arjona:2024cex}, because $fL$ is well within the regime where the exponential factors are not negligible. %as shown in Fig.\ \ref{fig:freq_dep}.
%\begin{figure}[ht]
%    \centering
%    \includegraphics[width=0.8\textwidth]{fig0.pdf}
%    \caption{The frequency-dependent ORF plotted as a function of $fL$. We see that for $fL$ near 1, the ORF is certainly not negligibly different from the frequency-independent value.}
%    \label{fig:freq_dep}
%\end{figure}
We used Monte-Carlo integration to numerically compute the frequency dependence of the effective ORF. The magnitude of $\Gamma_T(|f_{\text{min}}|)$ is significantly greater than $\Gamma_T$ with the factors of $\mathcal{E}(f, \hat{\bf \Omega})$ suppressed. If the ORF is in the regime where the Taylor expansions in the short wavelength ($fL \gg 1$) or long wavelength ($fL \ll 1$) cannot be carried out, then we must not ignore $\mathcal{E}(f, \hat{\bf \Omega})$.
\begin{figure}[ht]
    \centering
    \includegraphics[scale=0.48]{fig1.pdf}
    \caption{The ORFs plotted as a function of angular separation $\xi$. The solid lines use the full expression of the ORFs, including the factors of $\mathcal{E}$. The dotted lines are frequency independent and ignore the factors of $\mathcal{E}$. The solid lines are plotted with $fL = 1$}
    \label{fig:orfs}
\end{figure}
In Fig. \ref{fig:orfs}, we see how differently the ORFs behave when we take into account the frequency dependence and when we ignore it. We observe that for a fixed $fL$, the ratio $|\boldsymbol{k}|/k_0$ corresponds to an increasing disparity between the frequency-dependent and -independent ORFs. For sufficiently small values of the ratio $|\boldsymbol{k}|/k_0$, e.g.\ $|\boldsymbol{k}|/k_0 \lesssim 0.1$, the frequency-dependent ORF acquires additional local extrema, namely a minimum and a maximum. This is a peculiarity that only arises when factors of $\mathcal{E}$ are taken into account. We also expect such drastic behavior to emerge when the mass of the graviton is sufficiently high.

%\subsection{Graviton Mass}\label{subsec:mass}
The graviton mass is intimately connected to the behavior of the effective ORF. It appears as a relationship between the mass and the effective angular frequency. 
If observations of PTAs follow the trajectory for the best-case scenario, then we can only hope to observe a graviton mass given by that lower frequency bound. This corresponds to $m_g \sim 1.31\times 10^{-24} \eV$. To be clear, we are assuming that the lowest theoretical frequency that we can probe, given by the limit of the dispersion relation as $k\rightarrow 0$, will coincide with the lowest practical frequency that PTAs can measure, hence the ``best-case'' scenario. Going forth, this is the mass that we will take $m_g$ to be in our analysis.

The graviton mass affects all the ORFs for each polarization type, including the tensor modes. The graviton mass shows up in exponential terms as the coefficient in front of the dot product $\sqrt{1 - (m_g/k_0)^2}\hat{\bf \Omega}\cdot \hat{\boldsymbol{p}}_j$ and in the massive spin-1 helicity-0 polarization vector $\epsilon_\mu^0(m_g) = \sqrt{1 / (1 - (m_g/k_0)^2)}\times (\sqrt{1 - (m_g/k_0)^2},$ $\sin\theta\cos\varphi, \sin\theta\sin\varphi, \cos\theta)$. The former appears in all of the polarization types, but the latter only appears in the vector and scalar modes in the receiving functions, further complicating their mass dependences. The analytical expressions that we integrate can be found in Ref.~\cite{Liang:2021bct}, except that the expressions we use contain $\mathcal{E}$. Interestingly, because the graviton mass appears only as a ratio $m_g / k_0$, the behavior of the ORF is only sensitive to the mass relative to the frequency. In other words, there is no significance for an ORF purely due to a graviton mass of $m_g$. The question would be with regard to what frequency?

%\section{Results}\label{sec:results}
\textit{Results}---We now present the main results of this work. Although our calculations are for observations that extend well into the future, it is instructive to compare them to current data and confirm whether they are within the present standard deviations, or perhaps explain the current data better than HD. We compute the full frequency-dependent ORF using the same technique as earlier, i.e.\ Monte-Carlo integration. We keep $m_g$ fixed and modify the ratio $|\boldsymbol{k}|/k_0$ by changing the frequency $f$ we put in the ORF. We take two characteristic frequencies, one in the first half of the frequency interval (let us call this $f_0$), and one in the second half ($f_1$). This yields ratios of 0.1 and 0.5, respectively. We then compare the resulting effective ORFs with the PTA data that most closely resemble the ORF. For the ORF generated from $|\boldsymbol{k}|/k_0 = 0.1$, we compare it with the NANOGrav 15-year dataset \cite{Agazie:2023, Xu:2023wog}, which we will refer to hereafter as NANOGrav15, and for the ORF generated from $|\boldsymbol{k}|/k_0 = 0.5$, we compare it with the CPTA Data Release I (DR1) \cite{Xu:2023wog}.

We used NANOGrav15 to obtain angular correlations of 2,211 pairs of pulsars from a 67-pulsar array \cite{Agazie:2023}. 
\begin{figure}[ht]
    \centering
    \includegraphics[width=0.45\textwidth]{fig2.pdf}
    \caption{The frequency-dependent effective overlap reduction function plotted in red as a function of the angular separation $\xi$ between a pair of pulsars, in red. This is plotted at the upper frequency, $f_1$, and we set $|\boldsymbol{k}|/k_0 = 0.5$. The angular-separation–binned inter-pulsar correlations for NANOGrav15 are plotted with error bars.}
    \label{fig:ng}
\end{figure}
We do not use the frequentist optimal statistic that Ref.~\cite{Agazie:2023} employs, which is based on methods described in Ref.~\cite{Allen:2022ksj}, since it is crucial that the estimator is not biased towards the HD curve. We want the statistics to not assume anything about the additional polarizations, so that our comparison is sensible. For Fig.~\ref{fig:ng}, we used 13 bins constructed such that the mean of each bin matches the mean $\xi$'s of the CPTA data as closely as possible, since we are not able to alter the number or range of the bins in the CPTA data due to its public inaccessibility. Unfortunately, due to this complication, we computed the statistics for each binned average rather than the correlation coefficient for each pair, since we wanted the statistics for the two data sets to be on equal footing. 

We find that for NANOGrav15, HD gives a fit of $\chi^2$/d.o.f.\ $\sim 1.71$ whereas $\tilde{\Gamma}_T(f_1)$ with $|\boldsymbol{k}|/k_0 = 0.5$ gives a fit of $\chi^2$/d.o.f.\ $\sim 1.02$. The effective ORF from MG  is therefore a better fit overall, although it still suffers from not matching the data in the first half of the data.

We use CPTA DR1 to manually reconstruct the binned average values from 1,596 pairs of pulsars of a 57-pulsar array \cite{Xu:2023wog}. 
\begin{figure}[ht]
    \centering
    \includegraphics[width=0.45\textwidth]{fig3.pdf}
    \caption{The frequency-dependent effective overlap reduction function plotted as a function of the angular separation $\xi$ between a pair of pulsars, in blue. This is plotted at the lower bound of the frequency, $1/T_{\text{CTPA}}$, and we set $|\boldsymbol{k}|/k_0 = 0.1$. The angular–separation–binned inter-pulsar correlations for the CPTA dataset for $1/T_{\text{CTPA}}$ are plotted with error bars.}
    \label{fig:cpta}
\end{figure}
This data set is chosen because of the frequency dependence observed by CPTA \cite{Xu:2023wog}. CPTA has performed an analysis of the correlation coefficients for the frequencies $1/T_{\text{CTPA}}, 1.5/T_{\text{CTPA}},$ and $2/T_{\text{CTPA}}$, where $T_{\text{CTPA}} \sim 3.40$ years. We use the data for $1/T_{\text CTPA}$ since it is closer to $f_0$, which we wish to analyze. 

We find that for CPTA DR1, HD gives a fit of $\chi^2$/d.o.f.\ $\sim 3.00$ whereas $\tilde{\Gamma}_T(f_0)$ with $|\boldsymbol{k}|/k_0 = 0.1$ gives a fit of $\chi^2$/d.o.f.\ $\sim 1.51$. The effective ORF from MG  is a significantly better fit than HD. In fact, the effective ORF from MG does remarkably well in matching the data. We summarize our statistics in Table \ref{tbl:chi}.
\begin{table}[ht] 
\centering
\renewcommand{\arraystretch}{1.8}
\begin{tabular}{|c|c|c|c|}
\hline
\textbf{Data} & \textbf{Fit} & \textbf{$\chi^2$} & \textbf{$\chi^2$/d.o.f.} \\
\hline
NANOGrav15 & HD & 22.20 & 1.71 \\
\hline
NANOGrav15 & MG  & 11.22 & 1.02 \\
\hline
CPTA DR1 & HD & 38.95 & 3.00 \\
\hline
CPTA DR1 & MG  & 16.58 & 1.51 \\
\hline
\end{tabular}
\caption{The $\chi^2$ and $\chi^2$/d.o.f.\ values for different fit functions for the two datasets used in this analysis. We have 13 degrees of freedom for the HD correlation and 11 for the MG  models, with 2 fit parameters: $m_g$ and $f$. }
\label{tbl:chi}
\end{table}

%\section{Discussion}\label{sec:discussion}
\textit{Discussion and conclusion}---In this paper, we have reviewed the methods for obtaining the modified dispersion relation and effective ORF in the theory of ghost-free massive gravity. We have analyzed the behavior of this effective ORF when the exponential factors are not ignored. We were able to demonstrate that using unbiased NANOGrav15 and CPTA DR1, the effective ORF in MG fits the data significantly better than the HD curve with an optimistic graviton mass $m_g \sim$ $1.31\times10^{-24} \eV$.
%

In this Letter, we have not performed a rigorous fitting of the data; we simply used characteristic frequencies for comparison. A rigorous fitting to the full data in a different context has been done in Ref.~\cite{Arjona:2024cex}, and the ratio that has been found to best fit the frequentist optimal statistic of NANOGrav15 is $|\boldsymbol{k}|/k_0 \sim 0.73$. We deem this to be plausible on the basis of our statistics and the resulting shape of our ORF. A future study may introduce more free parameters than the ones we considered and perform a holistic fitting procedure to place constraints on the mass of the graviton and its relative energy densities. Additionally, it would be of use to perform the fitting on the individual pulsar-pulsar correlation coefficients rather than the binned averages if such data can be obtained for CPTA or other collaborations. 

An assumption we previously made is that the SGWB is isotropic. It is what allows us to decompose Eq.~\ref{eqn:two_point_z} as we did, but it is not necessarily an assumption that we can take for granted \cite{Depta:2024ykq, Bravo:2025csu, Cusin:2025xle, Kuwahara:2024jiz, Li:2024lvt}. Anisotropy in the background can be analyzed by decomposing the ORF on the basis of spherical harmonics and analyzing the multipole moments associated with these harmonics \cite{Allen:2024bnk, Gair:2014rwa}\footnote{It may not be appropriate to do spherical harmonic decomposition for the analysis of realistic PTAs \cite{Ali-Haimoud:2020ozu}. Instead, the Fischer formalism may be more effective, especially in the context of anisotropic backgrounds.}. 
%TK ask Murman about anisotropic Fisher analysis 
This can be applied to the effective ORFs we have derived and may be of interest in further work. 
%
The frequency dependence analysis of the ORF in general is underexplored. NANOGrav and other PTA collaborations do not claim to detect any frequency dependence in the ORF, while CPTA has. If the graviton is massive, then there will certainly be a frequency dependence in the data, even without considering $\mathcal{E}$. Doing an analysis of the frequency dependence of the NANOGrav data would, therefore, be quite elucidative. 

The detection prospects suggested in this Letter are quite optimistic. It may be unlikely that we would see a PTA experiment last that long, but we nonetheless provide a detailed overview of the ORFs that we may observe if such a scenario takes place. With regard to data from current PTA collaborations, the prospects seem hopeful of continuing to detect pulsar-pulsar correlation coefficients such that the ORFs generated in a theory of MG stay within their standard deviations. With more galactic pulsars being added to PTAs and the planned missions of space-based GW observatories (LISA and Taiji) on the horizon, it seems plausible that the mysteries that have evaded our investigations surrounding the cosmos may finally start to be unravelled. Clearly, a new era of GW observations is well underway. 

We conclude with the notion that these results do not depend on the origins of the SGWB. If gravitons are massive, then they will behave according to the principles laid out in this Letter, and we should expect to see the effects of this appear in various observables, regardless of whether the source is composed of continuous signals from supermassive black hole binaries or primordial GWs generated during inflation.

\vspace{5mm}
\textit{Acknowledgements}---We thank Qiuyue Liang, Neil Cornish, and Murman Gurgenidze for useful discussions related to the paper. We also thank the organizers of the 2025 Phenomenology Symposium (PHENO), during which much of this paper has been developed. C.C.\ and T.K.\ acknowledge support from the NASA Astrophysics Theory Program (ATP) Award 80NSSC22K0825 and the National Science Foundation (NSF) Astronomy and Astrophysics Research Grants (AAG) Award AST2408411.

\vspace{5mm}
\textit{Data availability}---Source code and data to reproduce all of our results (figures and Table \ref{tbl:chi}) are openly available \cite{Choi:2025git}.

\bibliographystyle{apsrev4-2_edited}
\bibliography{refs}


\clearpage
\end{document}
